\documentclass{beamer}
\usepackage{ctex}
\usepackage{graphicx}
\usepackage{float}

\usetheme{metropolis}
\title{代数式的值}
\subtitle{Value of Algebraic Expression}
\institute{Norsesun Milieu}
\author{K}
\date{\today}

\begin{document}
	\frame{\titlepage}
	
	\section{导入:代数式是什么?}
	\subsection{题目一}
	\begin{frame}{复习:列代数式}
		为了开展体育活动,学校要添置一批排球,每班配2个,学校另外
		留10个。已知学校共有n个班,问总共需要多少个排球?
		\pause
		
		(2n+10)个
	\end{frame}
	
	\begin{frame}{思考:}
			
				1.以上(2n+10)中的n表示什么?它可以取哪些数?
				
				2.学校有5个班,应添置多少个排球,如何求?
	\end{frame}
	
	\begin{frame}{结论:}
			
			当班数n取不同的值时,代数式 $2n+10$ 的计算结果也不同。
			即代数式的值随着n的改变而改变;只要给定n一个确定的值,
			代数式2n+10就有唯一确定的值与它对应。
	\end{frame}
	
	\section{何为值(Value)?}
	\begin{frame}{代数式的值(Value of algebraic expression)}
		一般的,如果把代数式里的字母用数代入,那么计算后得出得结果
		叫做代数式的值(Value of algebraic expression)。
		
		代数式里得字母可以取各种不同数值,但所取得数值必须使代数式
		和它表示得数量有实际意义。

		比如 $ \dfrac{s}{v} $ , v不能取零。
	\end{frame}
	
	\section{Practice}
	\begin{frame}{Practice A}
		某人买了50元的乘车月票卡,此人乘车次数用 m 表示,记录他每次					乘车后得余额用 n 表示,求:
		\begin{table}[H]
			\begin{tabular}{|rr|}
				\hline
					乘车次数 $m$ & 月卡余额 $n$(元)  \\
				\hline
					1 & 50-0.8  \\
					2 & 50-1.6 \\
					3 & 50-2.4 \\
					4 & 50-3.2 \\
					...& ... \\
				\hline
			
			\end{tabular}%
			%\qquad
			%($a^2 + b^2 = c^2$)
		\end{table}

			1. m与n之间得关系式。\pause \qquad
				$ n = 50 - 0.8m $
			
			2. 利用上述公式,计算乘了13次车还剩多少元?
				\pause \\
				解: 当m=13时,原式= $ 50 - 0.8\times13 = 39.6 $元
				
			3. 此人最多能乘几次车?
				\pause \\
				由题意可列不等式 $50 - 0.8m\geq0$,求得 $m\leq62.5$次
	\end{frame}
	
	\begin{frame}{Practice B}
		已知$ x^2 + 3x = -2$, 求多项式 $3x^2 + 9x -1 $的值。
	\end{frame}
	
\end{document}