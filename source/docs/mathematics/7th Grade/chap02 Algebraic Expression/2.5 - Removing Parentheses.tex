\documentclass{beamer}

\usepackage{ctex}
\usepackage{graphicx}
\usepackage{xcolor}
\usepackage{bm}

\usefonttheme[onlymath]{serif}% 数学公式字体设置

\usetheme[colorblocks]{Verona}
\title{去括号}
\subtitle{Removing Parentheses}
\author{K}
\date{最后更新:\today}
\logo{\includegraphics[height=1.2cm]{assets/Pngtree owl double exposure.png}}

\begin{document}
    \frame{\titlepage}

    \begin{frame}{括号前是加号 - 不变}{根据加法结合律去括号}
        \begin{alertblock}{加法结合律}
            $a + (b+c) = (a+b) +c$
        \end{alertblock}
        \begin{block}{}
            $ 7 + (9 + 1) =  \pause (7 + 9 ) +1 = 7 + 9 + 1 $
            $ 7 + (9 - 1) =  \pause 7 + [9 + (- 1)] = (7+9)+(-1)=7+9-1$
        \end{block}
        
    \end{frame}
    \begin{frame}{括号前是减号 - 变号}{理解括号前是减号的情况}
        \begin{columns}[t]
            \column{0.48\textwidth}
            \textbf{情景一} \quad 假设你课桌本来里有8本书,两本小说,三本漫画,三本教材。后来
            学校规定小说不能带,再后来学校规定漫画也不能带,你还可以带几本书?

            \begin{block}{}
               $ 8 - 2 - 3 = 3$
            \end{block}

            \column{0.48\textwidth}
            \textbf{情景二} \quad 假设你课桌本来里有8本书,两本小说,三本漫画,三本教材。后来
            学校规定小说和漫画不能带,你还可以带几本书?

            \begin{block}{}
                $8 - (2 + 3) = 8 - 2 - 3 = 3$
            \end{block}
        \end{columns}
        
        \vspace{5pt}
        
        \begin{block}{减去两个数的和,等于分别减去这两个数。}
        $8 - (-2 + 3) = \pause 8 -(-2)- 3 = 8 +2-3$\\
        $8 - (2-3) = \pause 8 - [2+(-3)]= 8- 2 - (-3)= 8 -2 + 3$
        \end{block}
    \end{frame}
    \begin{frame}{概括}{Overview}

            \begin{block}{规则}
                \begin{itemize}
                    \item 括号前是 “$\bm{+}$” 号,运用加法结合律把
                    括号和 它前面的“$\bm{+}$”
                    去掉, 原括号里各项的符号都\alert{不变}。
                    \item 括号前是 “$\bm{-}$”号,把括号和它前面的
                     “$\bm{-}$”都去掉,原括号里各项的符号都要\alert{改变}。
                \end{itemize}
            \end{block}


            \begin{columns}
                \column{0.49\textwidth}
                \begin{example}
                    \begin{itemize}
                        \item $a+(b+c) = a + b +c$
                        \item $a+(b-c) = a + b - c$ 
                        \item $a-(b-c) = a -b + c$ 
                        \item $a-(-b+c) = a + b -c $
                    \end{itemize}
                \end{example}

                \column{0.49\textwidth}
            \end{columns}
    \end{frame}

    \begin{frame}{另一种理解方式}{从乘法分配律的角度理解去括号}
                \begin{alertblock}{}
                    \begin{itemize}
                        \item $a+(b+c) = a + b +c$
                        \item $a+(b-c) = a + b - c$ 
                        \item $a-(b-c) = a -b + c$ 
                        \item $a-(-b+c) = a + b -c $
                    \end{itemize}
                \end{alertblock}

                \begin{block}{}
                    \begin{itemize}
                        \item $a+(b-c) = a + 1 \times (b +c)=a + b + c$
                        \item $a+(b-c) = a + 1 \times (b -c)=a + b - c$
                        \item $a-(b-c) = a + (-1) \times (b-c)=a+(-b)+c=a-b+c$ 
                    \end{itemize}
                \end{block}
    \end{frame}

    \begin{frame}{加括号 - Placing Parentheses}{代数的规则是双向的}
        \begin{alertblock}{}
            \begin{itemize}
                \item $a + b +c = a+(b+c)$
                \item $a + b - c = a+(b-c)$ 
                \item $ a -b + c= a-(b-c)$ 
                \item $ a + b -c= a-(-b+c) $
            \end{itemize}
        \end{alertblock}
    \end{frame}

    \begin{frame}{中括号与大括号}{Brackets and Braces}
        中括号 $[ \  ]$ 和大括号$\{ \}$和小括号的功能是一样的。为了书写清晰,我
        们用了小括号后用中括号,用了中括号后再用大括号。去括号的法则对中括号
        大括号也成立。
    \end{frame}

    \begin{frame}{练习}
        $8 + \{2 − [12 + (x − 2)]\}$

        \pause

        从里到外

        从外到里
    \end{frame}
\end{document}