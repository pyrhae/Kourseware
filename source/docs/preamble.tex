\documentclass[10pt]{ctexbeamer}
\usepackage{bm}
\usepackage{tikz}
\usepackage{amsmath}
\usepackage{graphicx}
\newfontfamily{\dengxian}{DengXian}
\newCJKfontfamily{\fzyaoti}{FZYaoTi} %方正姚体
\newCJKfontfamily{\fzjinghong}{FZZJ-JHTJW} %方正字迹-惊鸿体
\newCJKfontfamily{\dqheiti}{Hiragino Sans GB} %冬青黑体
\newCJKfontfamily{\fandolhei}{FandolHei}

% \usetheme[color blocks]{Verona}% 使用Verona主题
% \usetheme[color blocks, red]{Verona}% 使用Verona主题, red theme
% \usetheme[color blocks, gray]{Verona}% 使用Verona主题, grey theme
\usefonttheme[onlymath]{serif}% 数学公式字体设置
\author{Norsesun}
\date{最后更新:\today}
\logo{\includegraphics[height=1.2cm]{../../../Pngtree owl double exposure.png}}

\definecolor{airforceblue}{rgb}{.36,.54,.66}



\newcommand{\bmc}[1]{$\bm{#1}$}%定义一个新命令,行内数学模式的粗体,使幻灯片上的公式更清楚
\newcommand{\bmcc}[1]{
        \begin{displaymath}
            \bm{#1}
        \end{displaymath}
    }%定义一个新命令,行间数学模式的粗体,使幻灯片上的公式更清楚
\newcommand{\makecenter}[1]{\vspace{0.5em}\centering \parbox{.6\textwidth}{#1}}%定义一个新命令,居中排布一段话
\newenvironment{Mathbreakcenter}[1][1mm]{
        \par
        \vspace{#1} 
        \centering

    }{
        \par
        \vspace{2mm}
    }
\newcommand{\myblock}[3][1-]{
    \centering
    \begin{minipage}{.6\textwidth}
        \begin{block}<#1>{#2}%
            \centering%
            #3
        \end{block}
    \end{minipage}  
    } %定义一个新命令,居中排布一个block

    \newcommand{\myalertblock}[3][1-]{
        \centering
        \begin{minipage}{.6\textwidth}
            \begin{alertblock}<#1>{#2}%
                \centering%
                #3
            \end{alertblock}
        \end{minipage}  
        } %定义一个新命令,居中排布一个alertblock
\newcommand{\cleave}[2]{
    \hbox to #1{} #2 \hbox to #1{}
}
% \newcommand{\annmark}[1]{%
%     \textcolor{red}{$\bm\langle$#1$\bm\rangle$}%
% }%

% \newcommand{\ann}[1]{%
%     \begin{tikzpicture}[remember picture, baseline=-0.75ex]%
%         \node[coordinate] (inText) {};%
%     \end{tikzpicture}%
%     \marginpar{%
%         \renewcommand{\baselinestretch}{1.0}%
%         \begin{tikzpicture}[remember picture]%
%             \definecolor{orange}{rgb}{1,0.5,0}%
%             \draw node[fill=red!20,rounded corners,text width=\marginparwidth] (inNote){\footnotesize#1};%
%     \end{tikzpicture}%
%     }%
%     \begin{tikzpicture}[remember picture, overlay]%
%         \draw[draw = orange, thick]
%             ([yshift=-0.2cm] inText)
%                 -| ([xshift=-0.2cm] inNote.west)
%                 -| (inNote.west);%
%     \end{tikzpicture}%
% }%

% \setlength{\marginparwidth}{2.5cm}
% \renewcommand{\baselinestretch}{1.3}

\newenvironment{mathsalvation}[2][{解:}]{
    \begin{center}{}
        \begin{minipage}[t]{.05\textwidth}
            \vspace{0pt}
            {\color{#2}{#1}} \quad 
        \end{minipage}
        \begin{minipage}[t]{.7\textwidth}
            \vspace{0pt}
            % \fzyaoti
            % \dengxian
            % \fzjinghong
            % \dqheiti
            \fandolhei
}{
    \end{minipage}
    \end{center}
}
