% \documentclass{ctexart}
% % \usepackage{bm}
% \usepackage{tikz}

% % \usefonttheme[onlymath]{serif}% 数学公式字体设置

% \begin{document}


% \begin{tikzpicture}
%     \draw (0,0) circle (2cm);
%     \draw (2,0) circle (2cm);
%     % \clip[draw] (0,0) circle (2cm);
%     % \clip[draw] (2,0) circle (2cm);
%     \foreach \x in {-1,-0.75,-0.5,-0.25,0,0,
%         0.25,0.5,0.75,1,1.25,1.5,1.75,2,2.25,2.5,2.75}
%     \draw[xshift=\x cm]  (-2,2)--(2,-2);
%     \node at (1,0) {$I$};
% \end{tikzpicture}

% \raisebox{0pt}[0pt][0pt]{\Large%
% \textbf{Aaaa\raisebox{-0.3ex}{a}%
% \raisebox{-0.7ex}{aa}%
% \raisebox{-1.2ex}{r}%
% \raisebox{-2.2ex}{g}%
% \raisebox{-4.5ex}{h}}}
% he shouted but not even the next
% one in line noticed that something
% terrible had happened to him.

% \flushleft
% \newenvironment{vardesc}[1]{%
% \settowidth{\parindent}{#1:\ }
% \makebox[0pt][r]{#1:\ }}{}
% % \begin{displaymath}
% % a^2+b^2=c^2
% % \end{displaymath}
% \begin{vardesc}{Where}$a$,
%     $b$ -- are adjunct to the right
%     angle of a right-angled triangle.

%     $c$ -- is the hypotenuse of
%     the triangle and feels lonely.
    
%     $d$ -- finally does not show up
%     here at all. Isn’t that puzzling?
% \end{vardesc}

% \tikz \draw[thick,rounded corners=8pt]
% (0,0) -- (0,2) -- (1,3.25) -- (2,2) -- (2,0) -- (0,2) -- (2,2) -- (0,0) -- (2,0);

% xx
% \begin{tikzpicture}
%     \draw[line width=6pt](0, 0)circle(10pt);
% \end{tikzpicture}
% xx

% xx
% \begin{tikzpicture}
%     [baseline=(X.base)]
%     % \draw[line width=6pt](0, 0)circle(10pt);
%     \node [draw] (X) {world};
% \end{tikzpicture}
% xx

% Top align:
% \tikz[baseline=(current bounding box.north)]
% \draw (0,0) rectangle (1cm,1ex);

% \begin{tikzpicture}[execute at begin picture=%
%     { \draw (0,0) rectangle (2,2); }]
%     \node at (1,1) {\large X};
%     \end{tikzpicture}

%     \begin{tikzpicture}[execute at begin picture=%
%         { \draw (0,0) rectangle (2,2); }]
%         \node at (1,1) {\large X};
%         \end{tikzpicture}


% \end{document}


%\documentclass[14pt,handout]{beamer} %去除暂停打印
\documentclass[14pt]{beamer}
\usepackage{ctex}
\usepackage{makecell}
\usepackage{setspace}
\usepackage{ulem}


\usefonttheme[onlymath]{serif}% 数学公式字体设置

\usecolortheme[RGB={178,34,34}]{structure}% 耐火砖红
%\usecolortheme[RGB={205,173,0}]{structure}% 土金黄
\usetheme{Malmoe}

%\newtheorem{thm}{定理}%中文定理包
%\renewcommand\proofname{证明}%中文定理包


\setbeamertemplate{footline}{
     \begin{beamercolorbox}[sep=1ex]{author in head/foot}
     \rlap{\textit{\insertshorttitle}}\hfill\insertauthor\hfill\llap{\insertframenumber}%
     \end{beamercolorbox}%
}%设置页面底部带页码的infolines

\setbeamertemplate{items}[ball]%item 改为圆点
\setbeamertemplate{blocks}[rounded][shadow=true]% 更改定理边框
\setbeamertemplate{navigation symbols}{}% 去除底部工具导航栏

\newcommand {\p} {\par \vspace{1ex}}
\newcommand {\pp} {\par \vspace{1ex}\pause}
\newcommand {\red} {\textcolor[RGB]{178,34,34}}
\newcommand{\tk}[1][2.5]{\,\underline{\mbox{\hspace{#1 cm}}}\,}



\author{姓名}
\title{标题}
\subtitle{副标题}
\institute{}
\date{\today}


\begin{document}

\begin{frame}[plain]
  \titlepage
\end{frame}


\begin{frame}{}
  %\small{ %缩小字体
  \begin{spacing}{1.5}
    \textbf{例题 }  xxx
  \end{spacing}
  %}
\end{frame}

\begin{frame}{双栏格式}
  \small{
  \begin{spacing}{2.2}
  计算$\dfrac{3}{8} \div 3 \times \dfrac{3}{4}.$\pp

  以下解法是否正确?\pp
  \vspace{-2ex}

  \begin{columns}
  \begin{column}{0.5\textwidth}
   \begin{center}
    % \includegraphics[width=1\textwidth]{图2-8_1.jpg}
   \end{center}
  \end{column}\pause

  \begin{column}{0.6\textwidth}
    \begin{center}
    % \includegraphics[width=1\textwidth]{图2-8_2.jpg}
   \end{center}
  \end{column}
\end{columns}

  \end{spacing}
  }
\end{frame}


\begin{frame}{计算解答排版}
  \small{
  \begin{spacing}{2.2}
  \textbf{例题 }计算:\p
  \vspace{-2ex}
  (1)$\dfrac{12}{25}\div \dfrac{2}{3}\div \dfrac{6}{5}$; \hspace{5ex} (2)$\dfrac{3}{7}\times \dfrac{5}{6}+\dfrac{5}{14}\div \dfrac{2}{3}$.\pp

  \(
  \begin{aligned}
  \textbf{解: }(1)\hspace{1ex} \text{原式}= \pause &\dfrac{12}{25}\times \dfrac{3}{2}\times \dfrac{5}{6}\\ \pause
                                            = &\dfrac{3}{5}.\\ \pause
               (2)\hspace{1ex} \text{原式}= \pause &\dfrac{3}{7}\times \dfrac{5}{6}+\dfrac{5}{14}\times \dfrac{3}{2}\\ \pause
                                            = &\dfrac{5}{14}+\dfrac{15}{28}\\ \pause
                                            = &\dfrac{25}{28}.
  \end{aligned}
  \)

  \end{spacing}
  }
\end{frame}


\end{document}